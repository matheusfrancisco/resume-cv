%% The MIT License (MIT)
%%
%% Copyright (c) 2015 Daniil Belyakov
%%
%% Permission is hereby granted, free of charge, to any person obtaining a copy
%% of this software and associated documentation files (the "Software"), to deal
%% in the Software without restriction, including without limitation the rights
%% to use, copy, modify, merge, publish, distribute, sublicense, and/or sell
%% copies of the Software, and to permit persons to whom the Software is
%% furnished to do so, subject to the following conditions:
%%
%% The above copyright notice and this permission notice shall be included in all
%% copies or substantial portions of the Software.
%%
%% THE SOFTWARE IS PROVIDED "AS IS", WITHOUT WARRANTY OF ANY KIND, EXPRESS OR
%% IMPLIED, INCLUDING BUT NOT LIMITED TO THE WARRANTIES OF MERCHANTABILITY,
%% FITNESS FOR A PARTICULAR PURPOSE AND NONINFRINGEMENT. IN NO EVENT SHALL THE
%% AUTHORS OR COPYRIGHT HOLDERS BE LIABLE FOR ANY CLAIM, DAMAGES OR OTHER
%% LIABILITY, WHETHER IN AN ACTION OF CONTRACT, TORT OR OTHERWISE, ARISING FROM,
%% OUT OF OR IN CONNECTION WITH THE SOFTWARE OR THE USE OR OTHER DEALINGS IN THE
%% SOFTWARE.

% The font could be set to Windows-specific Calibri by using the 'calibri' option
\documentclass[]{mcdowellcv}

% For mathematical symbols
\usepackage{amsmath}
\usepackage{hyperref}

% Set applicant's personal data for header
\name{Matheus Francisco B. Machado}
\address{Rua Amaro José Pereira \linebreak Araranguá-SC}
\contacts{(048) 99149-6580 \linebreak matheusmachadoufsc@gmail.com}


\begin{document}

	% Print the header
	\makeheader
	
	% Print the content
	\begin{cvsection}{Empregos}
		\begin{cvsubsection}{Desenvolvedor de Software}{Empresa Júnior de Engenharia de Computação}{2016 -- 2017}
			~\\
			\begin{itemize}
				\item Membro de projetos: Atuou como desenvolvedor front e backend e também trabalho com desenvolvimendo de hardware.
				\item Membro do setor comercial: Realizava contato com os clientes e vendendo produtos.
				\item Membro de projetos: Capacitação de novos membros trainee.
			\end{itemize}
		\end{cvsubsection}
		
		\begin{cvsubsection}{Bolsista Iniciação Cientifica}{Universidade Federal de Santa Catarina}{2017 -- 2019}
			~\\
			\begin{itemize}
				\item Setor de pesquisa: realizou estudo sobre modelos de predição para classificar estudantes em risco de reprovação.
				\item Desenvolvimento: criação de modelos de predição para classificar estudantes em risco de reprovação utilizando Python.
				\item Desenvolvimento: criação de um dashboard buscando a integração do Moodle com modelos de predição para auxiliar os professores.
			\end{itemize}
		\end{cvsubsection}
		
		\begin{cvsubsection}{Professor de Física}{Universidade Federal de Santa Catarina}{ 2018 -- 2019}	
			~\\			
			\begin{itemize}
				\item Professor de Física no cursinho pré vestibular social da Universidade Federal de Santa Catarina.
			\end{itemize}
		\end{cvsubsection}
		
		
	\end{cvsection}
	
	\begin{cvsection}{Educação}
		\begin{cvsubsection}{Ararangua, SC}{Universaidade Federal de Santa Catarina}{ 2014 -- 2019}
			~\\
			\begin{itemize}
				\item Graduação: Bacharelado Engenharia de Computação. 
			\end{itemize}
		\end{cvsubsection}
	\end{cvsection}
	
	\begin{cvsection}{Experiência técnica}
		\begin{cvsubsection}{Projetos}{}{}
			\begin{itemize}
				\item \textbf{Fundador do projeto (2016)} Membro fundador Makerspace UFSC.
				\item \textbf{Professor (2016-2017)} Professor de lógica de programação para crianças no Arara Makerspace.
				\item \textbf{OSM MAP (2017)} Sistema de navegação utilizando Open Street Map e  algoritmos como  Dijkstra
				\item \textbf{Arm-Cortex3 Ext3 (2017)} Sistema de arquivo Ext3 para Cortex M3 ARM.
\item \textbf{Cafeteira crypto - (2018)} Integração entre Raspberry PI$^®$ com uma cafeteira, que aceitava pagamento com criptmoedas IOTA, utilizou: smart contract, JS, Python.
				\item \textbf{Real-Time-Embedded-Systems - Trabalho de graduação} (2018). Desenvolvimento de um sistema operacional para o PIC18f4520.
				\item \textbf{Professor (2018-2019)} Professor de programação em Arduino no Arara Makerspace.
				\item \textbf{Professor (2018-2019)} Professor de Física no cursinho Social da UFSC.
				\item \textbf{ExgMachine (2018-2019)} Trabalho de conclusão de curso sistema de compra utilizando pagamento com criptmoedas, realizando integração hardware e software.
			\end{itemize}
		\end{cvsubsection}
	\end{cvsection}
	
	
	\begin{cvsection}{Habilidades}
		\begin{cvsubsection}{}{}{}	
			\begin{itemize}
				\item Python (Flask, Django, scikit-learn, pandas, NumPy)
				\item Javascript (NodeJS, Express, CoffeScript, Socket.io, JQuery, AJAX, Grunt, VueJS)
				\item Java				
				\item C/C++
				\item Database (PostgreSQL, MySQL, MongoDB)
				\item Google Cloud Plataform (Firebase), Heroku
				\item Docker
				\item Git
				\item REST-API
				\item Postman, Insomia
				\item Blockchain, Smart Contract (Solidity)
				 
			\end{itemize}
		\end{cvsubsection}
	\end{cvsection}
	\begin{cvsection}{Estudando sobre}
		\begin{cvsubsection}{}{}{}	
			\begin{itemize}
				\item Clojure
				\item Rust
				\item Design Patterns				
				\item Domain Driven Design
				\item Database (Datonic)
				\item ReactJS, React Native
				\item Blockchain, Smart Contract (Solidity)
				 
			\end{itemize}
		\end{cvsubsection}
	\end{cvsection}

	\begin{cvsection}{Links}
		\begin{cvsubsection}{}{}{}	
			\begin{itemize}
				\item Website: \url{matheusfrancisco.github.io}
				\item Arara Makerspace:  \url{araramakespace.github.io}
				\item Github: \url{https://github.com/matheusfrancisco}
				\item Linkedin: \url{https://www.linkedin.com/in/matheus-francisco} 
			\end{itemize}
		\end{cvsubsection}
	\end{cvsection}
	
\end{document}
