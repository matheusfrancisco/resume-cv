%% The MIT License (MIT)
%%
%% Copyright (c) 2015 Daniil Belyakov
%%
%% Permission is hereby granted, free of charge, to any person obtaining a copy
%% of this software and associated documentation files (the "Software"), to deal
%% in the Software without restriction, including without limitation the rights
%% to use, copy, modify, merge, publish, distribute, sublicense, and/or sell
%% copies of the Software, and to permit persons to whom the Software is
%% furnished to do so, subject to the following conditions:
%%
%% The above copyright notice and this permission notice shall be included in all
%% copies or substantial portions of the Software.
%%
%% THE SOFTWARE IS PROVIDED "AS IS", WITHOUT WARRANTY OF ANY KIND, EXPRESS OR
%% IMPLIED, INCLUDING BUT NOT LIMITED TO THE WARRANTIES OF MERCHANTABILITY,
%% FITNESS FOR A PARTICULAR PURPOSE AND NONINFRINGEMENT. IN NO EVENT SHALL THE
%% AUTHORS OR COPYRIGHT HOLDERS BE LIABLE FOR ANY CLAIM, DAMAGES OR OTHER
%% LIABILITY, WHETHER IN AN ACTION OF CONTRACT, TORT OR OTHERWISE, ARISING FROM,
%% OUT OF OR IN CONNECTION WITH THE SOFTWARE OR THE USE OR OTHER DEALINGS IN THE
%% SOFTWARE.

% The font could be set to Windows-specific Calibri by using the 'calibri' option
\documentclass[]{mcdowellcv}

% For mathematical symbols
\usepackage{amsmath}
\usepackage{hyperref}

% Set applicant's personal data for header
\name{Matheus Francisco B. Machado}
\address{Rua amaro josé pereira \linebreak Araranguá-SC}
\contacts{(048) 99149-6580 \linebreak matheusmachadoufsc@gmail.com}


\begin{document}

	% Print the header
	\makeheader
	
	% Print the content
	\begin{cvsection}{Employment}
		\begin{cvsubsection}{Desenvolvedor de Software}{Empresa Júnior de Engenharia de Computação}{2016 -- 2017}
			~\\
			\begin{itemize}
				\item Membro de projetos: Atuou como desenvolvedor front e backend e também trabalho com desenvolvimendo de hardware.
				\item Membro do setor comercial: Realizava contato com os clientes e vendendo produtos.
				\item Membro de projetos: Capacitação de novos membros trainee
			\end{itemize}
		\end{cvsubsection}
		
		\begin{cvsubsection}{Bolsista Iniciação Cientifica}{Universidade Federal de Santa Catarina}{2017 -- 2019}
			~\\
			\begin{itemize}
				\item Setor de pesquisa: realizou estudo sobre modelos de predição para classificar estudantes em risco de reprovação
				\item Desenvolvimento: criação de modelos de predição para classificar estudantes em risco de reprovação utilizando python
				\item Desenvolvimento: criação de um dashboard buscando a integração do moodle com modelos de predição para auxiliar os professores.
			\end{itemize}
		\end{cvsubsection}
		
		\begin{cvsubsection}{Professor de Física}{Universidade Federal de Santa Catarina}{ 2018 -- 2019}	
			~\\			
			\begin{itemize}
				\item Professor de Física no cursinho pré vestibular social da Universidade Federal de Santa Catarina.
			\end{itemize}
		\end{cvsubsection}
		
		
	\end{cvsection}
	
	\begin{cvsection}{Education}
		\begin{cvsubsection}{Ararangua, SC}{Universaidade Federal de Santa Catarina}{ 2014 -- 2019}
			~\\
			\begin{itemize}
				\item Graduação: Bacharelado Engenharia de Computação 
			\end{itemize}
		\end{cvsubsection}
	\end{cvsection}
	
	\begin{cvsection}{Experiência técnica}
		\begin{cvsubsection}{Projetos}{}{}
			\begin{itemize}
				\item \textbf{Cafeteira crypto - Trabalho de graduação} (2018). Integração entre raspberry pi com uma cafeteira, que aceitava pagamento com criptmoedas IOTA, utilizou: smart contract, js, python
				\item \textbf{Real-Time-Embedded-Systems - Trabalho de graduação} (2018). Desenvolvimento de um sistema operacional para o  PIC18f4520
				\item \textbf{OSM MAP} (2017).  Sistema de navegação utilizando Open street map e  algoritmos como  dijkstra
				\item \textbf{Arm-Cortex3 Ext3} (2017).  Sistema de arquivo ext3 para Cortex M3 ARM
				\item \textbf{ExgMachine} (2018-2019).  Trabalho de conclusão de curso sistema de compra utilizando pagamento com criptmoedas, realizando integração hardware e software
				\item \textbf{Professor (2017-2018)} Professor de Física no cursinho Social da UFSC;
				\item \textbf{Fundador do projeto} Membro fundador Makerspace UFSC .
				\item \textbf{Professor (2016-2017)} Professor de lógica de programação para crianças no Arara makerspace;
				\item \textbf{Professor (2018-2019)} Professor de programação em Arduino no Arara makerspace;
			\end{itemize}
		\end{cvsubsection}
	\end{cvsection}
	
	
	\begin{cvsection}{Languages and Technologies}
		\begin{cvsubsection}{}{}{}	
			\begin{itemize}
				\item Python; Javascript; C; SQL; MongoDB; JavaScript; Java;
				\item Flask, Nodejs; 
			\end{itemize}
		\end{cvsubsection}
	\end{cvsection}

	\begin{cvsection}{Links}
		\begin{cvsubsection}{}{}{}	
			\begin{itemize}
				\item website: \url{matheusfrancisco.github.io}
				\item araramaker :  \url{araramakespace.github.io}
				\item github: \url{https://github.com/matheusfrancisco}
				\item linkedin: \url{https://www.linkedin.com/in/matheus-francisco/} 
			\end{itemize}
		\end{cvsubsection}
	\end{cvsection}
	
\end{document}
